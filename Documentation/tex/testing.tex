\documentclass[12pt]{article}
\usepackage[frenchb]{babel}
\usepackage[utf-8]{inputenc}  %%latin1
\usepackage{fullpage}
\usepackage{graphicx}		%% to insert pics
\usepackage{listings}
\usepackage{color}
\usepackage{url}
\usepackage{pythonhighlight}
\usepackage{hyperref}
\usepackage{xcolor}

\title{Testing}

\hypersetup{linkcolor=blue}
\definecolor{medium-blue}{rgb}{0,0,139}
\definecolor{mygreen}{rgb}{0,0.6,0}
\definecolor{mygray}{rgb}{0.5,0.5,0.5}
\definecolor{mymauve}{rgb}{0.58,0,0.82}
\hypersetup{colorlinks, urlcolor={medium-blue}}
\date{}
\lstset{ %
  backgroundcolor=\color{white},   % choose the background color
  basicstyle=\footnotesize,        % size of fonts used for the code
  breaklines=true,                 % automatic line breaking only at whitespace
  captionpos=b,                    % sets the caption-position to bottom
  commentstyle=\color{mygreen},    % comment style
  escapeinside={\%*}{*)},          % if you want to add LaTeX within your code
  keywordstyle=\color{blue},       % keyword style
  stringstyle=\color{mymauve},     % string literal style
}


\begin{document}
\maketitle
\begin{center}

Im Ordner TestRoutine befindet sich eine File \color{mymauve}test.py\color{black}.
Um das Projekt richtig testen zu können, wurde die \color{mymauve}test.py \color{black} so eingerichtet, dass beliebig
viele Playlists angesprochen werden können. 
\end{center}
\hline \vspace{0.2cm}
\color{mymauve}
Benötigte libraries:
\color{black}
\begin{itemize}
    \item pandas
    \item mlxtend
    \item tqdm
    \item psycopg2
    \item json
\end{itemize}\vspace{0.2cm}
\hline \vspace{0.2cm}
\\\\
Auch das Überprüfen einer validen submission wird übernommen. Dies funktioniert allerdings nur bei einer Angabe
von \color{red}10, 20 oder 30 \color{black} Playlists.
\\\\
\hline \vspace{0.2cm}
\color{mymauve}Es muss ein Tunnel zur Datenbank existieren!\color{black}\vspace{0.2cm}
\hline \vspace{0.2cm}

\begin{center}
\begin{python}
python test.py 10  |   python test.py 20  |  python test.py 30  
\end{python}
\end{center}
\\
Zunächst wird eine \color{mymauve}test.csv \color{black} erstellt, in welcher die relevanten Playlistdaten aus der Datenbank zwischengespeichert werden.
\\\\Nun werden die Daten per Logik, welche aus der RuleMining stammt, in Regeln abgewandelt. Diese werden dann in der \color{mymauve}testrules.csv \color{black} zwischengespeichert.
\\\\Letztlich werden die Regeln noch angewendet, und eine endgültige File testsubmissions.csv erstellt.

\end{document}